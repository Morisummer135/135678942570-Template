\section{Geometry}
\large


\subsection{注意}
\begin{enumerate}[I. ]
	\item 注意舍入方式($0.5$的舍入方向);防止输出$-0$.
	\item 几何题注意多测试不对称数据.
	\item 整数几何注意xmult和dmult是否会出界;\\
	      符点几何注意eps的使用.
	\item 避免使用斜率;注意除数是否会为$0$.
	\item 公式一定要化简后再代入.
	\item 判断同一个$2\times PI$域内两角度差应该是\\
   		  $abs(a1-a2)<beta \parallel abs(a1-a2)>\pi+\pi-beta;$\\
          相等应该是\\
          $abs(a1-a2)<eps \parallel abs(a1-a2)>\pi+\pi-eps.$
    \item 需要的话尽量使用$atan2$,注意:$atan2(0,0)=0$,\\
          $atan2(1,0)=\pi/2$,$atan2(-1,0)=-\pi/2$,$atan2(0,1)=0$,$atan2(0,-1)=\pi$.
    \item cross product = $\mid u \mid\times \mid v\mid\times sin(a)$\\
          dot product = $\mid u \mid\times \mid v\mid\times cos(a)$
    \item $(P1-P0)X(P2-P0)$结果的意义:\\
          正: $<P0,P1>$在$<P0,P2>$顺时针$(0,\pi)$内\\
          负: $<P0,P1>$在$<P0,P2>$逆时针$(0,\pi)$内\\
          0 : $<P0,P1>$,$<P0,P2>$共线,夹角为$0$或$\pi$
    \item 误差限缺省使用$1e-8$!
\end{enumerate}


\subsection{几何公式}
\subsubsection{三角形}
\begin{enumerate}[I. ]
	\item 半周长 $P=\frac{a+b+c}{2}$
	\item 面积 $S=\frac{a\times H}{2}=\frac{a\times b\times sin(C)}{2}=\sqrt{P\times (P-a)\times (P-b)\times (P-c)}$
	\item 中线 $Ma=\frac{\sqrt{2\times (b^{2}+c^{2})-a^{2}}}{2}=\frac{\sqrt{b^{2}+c^{2}+2\times b\times c\times cos(A)}}{2}$
	\item 角平分线 $Ta=\frac{\sqrt{b\times c((b+c)^{2}-a^{2})}}{b+c}=\frac{2\times b\times c\times cos(\frac{A}{2})}{b+c}$
	\item 高线 $Ha=b\times sin(C)=c\times sin(B)=\sqrt{b^{2}-(\frac{a^{2}+b^{2}-c^{2}}{2\times a})^{2}}$
	\item 内切圆半径 $r=\frac{S}{P}=\frac{a\times sin(\frac{B}{2})\times sin(\frac{C}{2})}{sin(\frac{B+C}{2})}$\\
               $=4\times R\times sin(\frac{A}{2})\times sin(\frac{B}{2})\times sin(\frac{C}{2})=\sqrt{\frac{(P-a)\times (P-b)\times (P-c)}{P}}$\\
               $=P\times tan(\frac{A}2)\times tan(\frac{B}2)\times tan(\frac{C}2)$
    \item 外接圆半径 $R=\frac{a\times b\times c}{4\times S}=\frac{a}{2\times sin(A)}=\frac{b}{2\times sin(B)}=\frac{c}{2\times sin(C)}$
\end{enumerate}
\subsubsection{四边形}
$D1$,$D2$为对角线,$M$对角线中点连线,$A$为对角线夹角
\begin{enumerate}[I. ]
	\item $a^{2}+b^{2}+c^{2}+d^{2}=D1^{2}+D2^{2}+4\times M^{2}$
	\item $S=\frac{D1\times D2\times sin(A)}2$\\
\end{enumerate}
\subsubsection{圆内接四边形}
\begin{enumerate}[I. ]
	\item $a\times c+b\times d=D1\times D2$
	\item $S=\sqrt{(P-a)\times (P-b)\times (P-c)\times (P-d)}$,$P$为半周长
\end{enumerate}
\subsubsection{正$N$边形}
$R$为外接圆半径,$r$为内切圆半径
\begin{enumerate}
	\item 中心角 $A=\frac{2\times \pi}{N}$
	\item 内角 $C=\frac{(N-2)\times \pi}{N}$
	\item 边长 $a=2\times \sqrt{R^{2}-r^{2}}=2\times R\times sin(\frac{A}{2})=2\times r\times tan(\frac{A}{2})$
	\item 面积 $S=\frac{N\times a\times r}{2}=N\times r^{2}\times tan(\frac{A}{2})=\frac{N\times R^{2}\times sin(A)}{2}=\frac{N\times a^{2}}{4\times tan(\frac{A}{2})}$
\end{enumerate}
\subsubsection{圆}
\begin{enumerate}[I. ]
	\item 弧长 $l=rA$
	\item 弦长 $a=2\times \sqrt{2\times h\times r-h^{2}}=2\times r\times sin(\frac{A}{2})$
	\item 弓形高 $h=r-\sqrt{r^{2}-\frac{a^{2}}{4}}=r\times (1-cos(\frac{A}{2}))=\frac{a\times tan(\frac{A}{4})}{2}$
	\item 扇形面积 $S1=\frac{r\times l}{2}=\frac{r^{2}\times A}2$
	\item 弓形面积 $S2=\frac{r\times l-a\times(r-h)}{2}=\frac{r^{2}\times (A-sin(A))}{2}$
\end{enumerate}
\subsubsection{棱柱}
\begin{enumerate}[I. ]
	\item 体积 $V=A\times h$ $A$为底面积,$h$为高
	\item 侧面积 $S=l\times p$ $l$为棱长,$p$为直截面周长
	\item 全面积 $T=S+2\times A$
\end{enumerate}
\subsubsection{棱锥}
\begin{enumerate}[I. ]
	\item 体积 $V=\frac{A\times h}{3}$ $A$为底面积,$h$为高
\end{enumerate}
\subsubsection{正棱锥}
\begin{enumerate}[I. ]
	\item 侧面积 $S=\frac{l\times p}{2}$ $l$为斜高,$p$为底面周长
	\item 全面积 $T=S+A$
\end{enumerate}
\subsubsection{棱台}
\begin{enumerate}[I. ]
	\item 体积 $V=\frac{(A1+A2+\sqrt{A1\times A2})\times h}{3}$ $A1$,$A2$为上下底面积,$h$为高
\end{enumerate}
\subsubsection{正棱台}
\begin{enumerate}[I. ]
	\item 侧面积 $S=\frac{(p1+p2)\times l}{2}$ $p1$,$p2$为上下底面周长,$l$为斜高
	\item 全面积 $T=S+A1+A2$
\end{enumerate}
\subsubsection{圆柱}
\begin{enumerate}[I. ]
	\item 侧面积 $S=2\times \pi \times r \times h$
	\item 全面积 $T=2\times \pi \times r \times (h+r)$
	\item 体积 $V=\pi \times r^{2} \times h$
\end{enumerate}
\subsubsection{圆锥}
\begin{enumerate}[I. ]
	\item 母线 $l=\sqrt{h^{2}+r^{2}}$
	\item 侧面积 $S=\pi\times r\times l$
	\item 全面积 $T=\pi\times r\times (l+r)$
	\item 体积 $V=\frac{\pi\times r^{2}\times h}{3}$
\end{enumerate}
\subsubsection{圆台}
\begin{enumerate}[I. ]
	\item 母线 $l=\sqrt{h^{2}+(r1-r2)^{2}}$
	\item 侧面积 $S=\pi\times (r1+r2)\times l$
	\item 全面积 $T=\pi\times r1\times (l+r1)+\pi\times r2\times (l+r2)$
	\item 体积 $V=\frac{\pi\times (r1^{2}+r2^{2}+r1\times r2)\times h}{3}$
\end{enumerate}
\subsubsection{球}
\begin{enumerate}[I. ]
	\item 全面积 $T=4\times \pi \times r^{2}$
	\item 体积 $V=\frac{4\times \pi\times r^{3}}{3}$
\end{enumerate}
\subsubsection{球台}
\begin{enumerate}[I. ]
	\item 侧面积 $S=2\times \pi\times r\times h$
	\item 全面积 $T=\pi\times (2\times r\times h+r1^{2}+r2^{2})$
	\item 体积 $V=\frac{\pi\times h\times (3\times (r1^{2}+r2^{2})+h^{2})}{6}$
\end{enumerate}
\subsubsection{球扇形}
\begin{enumerate}[I. ]
	\item 全面积 $T=\pi\times r\times (2\times h+r0)$ $h$为球冠高,$r0$为球冠底面半径
	\item 体积 $V=\frac{2\times \pi\times r^{2}\times h}{3}$
\end{enumerate}

\subsection{多边形}
\subsubsection{头文件}
\lstinputlisting{"./geometry/polygon/head.cpp"}
\subsubsection{判定凸多边形,允许相邻边共线}
\lstinputlisting{"./geometry/polygon/is_convex.cpp"}
\subsubsection{判定凸多边形,不允许相邻边共线}
\lstinputlisting{"./geometry/polygon/is_convex_v2.cpp"}
\subsubsection{判点在凸多边形内或多边形边上}
\lstinputlisting{"./geometry/polygon/inside_convex.cpp"}
\subsubsection{判点在凸多边形内}
\lstinputlisting{"./geometry/polygon/inside_convex_v2.cpp"}
\subsubsection{判点在任意多边形内}
\lstinputlisting{"./geometry/polygon/inside_polygon.cpp"}
\subsubsection{判线段在任意多边形内}
\lstinputlisting{"./geometry/polygon/inside_polygon_v2.cpp"}
\subsubsection{多边形重心}
\lstinputlisting{"./geometry/polygon/barycenter.cpp"}

\subsection{浮点函数}
\subsubsection{头文件}
\lstinputlisting{"./geometry/float_function/head.cpp"}
\subsubsection{两点距离}
\lstinputlisting{"./geometry/float_function/dis.cpp"}
\subsubsection{判三点共线}
\lstinputlisting{"./geometry/float_function/dots_inline.cpp"};
\subsubsection{判点在线段上,包括端点}
\lstinputlisting{"./geometry/float_function/dot_online_in.cpp"}
\subsubsection{判点在线段上,不包括端点}
\lstinputlisting{"./geometry/float_function/dot_online_ex.cpp"}
\subsubsection{判两点在线段同侧,点在线段上返回0}
\lstinputlisting{"./geometry/float_function/same_side.cpp"}
\subsubsection{判两点在线段异侧,点在线段上返回0}
\lstinputlisting{"./geometry/float_function/opposite_side.cpp"}
\subsubsection{判两直线平行}
\lstinputlisting{"./geometry/float_function/parallel.cpp"}
\subsubsection{判两直线垂直}
\lstinputlisting{"./geometry/float_function/perpendicular.cpp"}
\subsubsection{判两线段相交,包括端点和部分重合}
\lstinputlisting{"./geometry/float_function/intersect_in.cpp"}
\subsubsection{判两线段相交,不包括端点和部分重合}
\lstinputlisting{"./geometry/float_function/intersect_ex.cpp"}
\subsubsection{计算两直线交点}
\lstinputlisting{"./geometry/float_function/intersection.cpp"}
\subsubsection{点到直线上的最近点}
\lstinputlisting{"./geometry/float_function/ptoline.cpp"}
\subsubsection{点到直线距离}
\lstinputlisting{"./geometry/float_function/disptoline.cpp"}
\subsubsection{点到线段上的最近点}
\lstinputlisting{"./geometry/float_function/ptoseg.cpp"}
\subsubsection{点到线段距离}
\lstinputlisting{"./geometry/float_function/disptoseg.cpp"}
\subsubsection{矢量V以P为顶点逆时针旋转angle并放大scale倍}
\lstinputlisting{"./geometry/float_function/rotate.cpp"}

\subsection{三角形}
\subsubsection{头文件}
\lstinputlisting{"./geometry/triangle/head.cpp"}
\subsubsection{外心}
\lstinputlisting{"./geometry/triangle/circumcenter.cpp"}
\subsubsection{内心}
\lstinputlisting{"./geometry/triangle/incenter.cpp"}
\subsubsection{垂心}
\lstinputlisting{"./geometry/triangle/perpencenter.cpp"}
\subsubsection{重心}
\lstinputlisting{"./geometry/triangle/barycenter.cpp"}
\subsubsection{费马点}
\lstinputlisting{"./geometry/triangle/fermentpoint.cpp"}

\subsection{三维几何}
\subsubsection{头文件}
\lstinputlisting{"./geometry/3Dgeometry/head.cpp"}
\subsubsection{取平面法向量}
\lstinputlisting{"./geometry/3Dgeometry/pvec.cpp"}
\subsubsection{判三点共线}
\lstinputlisting{"./geometry/3Dgeometry/dots_inline.cpp"}
\subsubsection{判四点共面}
\lstinputlisting{"./geometry/3Dgeometry/dots_onplane.cpp"}
\subsubsection{判点是否在线段上,包括端点和共线}
\lstinputlisting{"./geometry/3Dgeometry/dot_online_in.cpp"}
\subsubsection{判点是否在线段上,不包括端点}
\lstinputlisting{"./geometry/3Dgeometry/dot_online_ex.cpp"}
\subsubsection{判点是否在空间三角形上,包括边界,三点共线无意义}
\lstinputlisting{"./geometry/3Dgeometry/dot_inplane_in.cpp"}
\subsubsection{判点是否在空间三角形上,不包括边界,三点共线无意义}
\lstinputlisting{"./geometry/3Dgeometry/dot_inplane_ex.cpp"}
\subsubsection{判两点在线段同侧,点在线段上返回0,不共面无意义}
\lstinputlisting{"./geometry/3Dgeometry/same_side.cpp"}
\subsubsection{判两点在线段异侧,点在线段上返回0,不共面无意义}
\lstinputlisting{"./geometry/3Dgeometry/opposite_side.cpp"}
\subsubsection{判两点在平面同侧,点在平面上返回0}
\lstinputlisting{"./geometry/3Dgeometry/same_side2.cpp"}
\subsubsection{判两点在平面异侧,点在平面上返回0}
\lstinputlisting{"./geometry/3Dgeometry/opposite_side2.cpp"}
\subsubsection{判两直线平行}
\lstinputlisting{"./geometry/3Dgeometry/parallel.cpp"}
\subsubsection{判两平面平行}
\lstinputlisting{"./geometry/3Dgeometry/parallel2.cpp"}
\subsubsection{判直线与平面平行}
\lstinputlisting{"./geometry/3Dgeometry/parallel3.cpp"}
\subsubsection{判两直线垂直}
\lstinputlisting{"./geometry/3Dgeometry/perpendicular.cpp"}
\subsubsection{判两平面垂直}
\lstinputlisting{"./geometry/3Dgeometry/perpendicular2.cpp"}
\subsubsection{判直线与平面垂直}
\lstinputlisting{"./geometry/3Dgeometry/perpendicular3.cpp"}
\subsubsection{判两线段相交,包括端点和部分重合}
\lstinputlisting{"./geometry/3Dgeometry/intersect_in.cpp"}
\subsubsection{判两线段相交,不包括端点和部分重合}
\lstinputlisting{"./geometry/3Dgeometry/intersect_ex.cpp"}
\subsubsection{判线段与空间三角形相交,包括交于边界和(部分)包含}
\lstinputlisting{"./geometry/3Dgeometry/intersect_in2.cpp"}
\subsubsection{判线段与空间三角形相交,不包括交于边界和(部分)包含}
\lstinputlisting{"./geometry/3Dgeometry/intersect_ex2.cpp"}
\subsubsection{计算两直线交点}
\lstinputlisting{"./geometry/3Dgeometry/intersection.cpp"}
\subsubsection{计算直线与平面交点}
\lstinputlisting{"./geometry/3Dgeometry/intersection2.cpp"}
\subsubsection{计算两平面交线}
\lstinputlisting{"./geometry/3Dgeometry/intersection3.cpp"}
\subsubsection{点到直线距离}
\lstinputlisting{"./geometry/3Dgeometry/ptoline.cpp"}
\subsubsection{点到平面距离}
\lstinputlisting{"./geometry/3Dgeometry/ptoplane.cpp"}
\subsubsection{直线到直线距离}
\lstinputlisting{"./geometry/3Dgeometry/linetoline.cpp"}
\subsubsection{两直线夹角cos值}
\lstinputlisting{"./geometry/3Dgeometry/angle_cos.cpp"}
\subsubsection{两平面夹角cos值}
\lstinputlisting{"./geometry/3Dgeometry/angle_cos2.cpp"}
\subsubsection{直线平面夹角sin值}
\lstinputlisting{"./geometry/3Dgeometry/angle_sin.cpp"}

\subsection{网格}
\lstinputlisting{"./geometry/grids.cpp"}

\subsection{圆}
\subsubsection{头文件}
\lstinputlisting{"./geometry/circle/head.cpp"}
\subsubsection{判直线和圆相交,包括相切}
\lstinputlisting{"./geometry/circle/intersect_line_circle.cpp"}
\subsubsection{判线段和圆相交,包括端点和相切}
\lstinputlisting{"./geometry/circle/intersect_seg_circle.cpp"}
\subsubsection{判圆和圆相交,包括相切}
\lstinputlisting{"./geometry/circle/intersect_circle_circle.cpp"}
\subsubsection{计算圆上到点p最近点,如p与圆心重合,返回p本身}
\lstinputlisting{"./geometry/circle/dot_to_circle.cpp"}
\subsubsection{计算直线与圆的交点}
\lstinputlisting{"./geometry/circle/intersection_line_circle.cpp"}
\subsubsection{计算圆与圆的交点}
\lstinputlisting{"./geometry/circle/intersection_circle_circle.cpp"}

\subsection{凸包重心}
\lstinputlisting{"./geometry/convex.cpp"}

\subsection{半平面交}
\lstinputlisting{"./geometry/HPI.cpp"}

\subsection{圆面积并}
\lstinputlisting{"./geometry/circle_area_add.cpp"}

\subsection{圆与多边形交}
\lstinputlisting{"./geometry/circle_polygon_area.cpp"}

\subsection{三维凸包}
\lstinputlisting{"./geometry/convex3d.cpp"}
